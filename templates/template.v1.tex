\documentclass[11pt,a4paper,oneside]{article}

\usepackage{amsmath}
\usepackage{amssymb}
\usepackage{geometry}  % 宏包]
\usepackage[UTF8]{ctex}
\usepackage{xcolor}

\geometry{left=1.8cm,right=1.8cm,top=1.5cm,bottom=2cm}
%设置页边距
\geometry{centering}
%版心水平垂直均居中,竖直居中:vcentering,水平居中:hcenterin

\usepackage{fancyhdr}

\linespread{1.25}  % 行距
\usepackage{fancyhdr}  % 宏包
\usepackage{lastpage}  % 获得总页数宏包
\pagestyle{fancy}
\fancyhf{}  % 清除页眉页脚保证默认样式不起作用

\rhead{杰哥讲数学,抖音搜索 sqrt2\_1.41421}
\usepackage{fontspec}  % 宏包



%题目
\newfontfamily\timuen{Times New Roman}   % 设置英文局部字体
\newCJKfontfamily\timucn{simhei.ttf}  % 设置中文局部字体 黑体
\newcommand{\timu}[1]{\timucn \timuen #1}  % 混合中英文
%作者:
\newfontfamily\zzen{Times New Roman}  % 设置英文局部字体
%\newCJKfontfamily\zzcn{simsong.ttf}  % 设置中文局部字体 仿宋
\newCJKfontfamily\zzcn{simfang.ttf}  % 设置中文局部字体 仿宋
\newcommand{\zz}[1]{\zzcn \zzen #1}  % 混合中英文
%单位:
\newfontfamily\dwen{Times New Roman}  % 设置英文局部字体
\newCJKfontfamily\dwcn{simsun.ttc}  % 设置中文局部字体 宋体
\newcommand{\dw}[1]{\dwcn \dwen #1}  % 混合中英文
%摘要和关键字:
\newfontfamily\zygjzten{Times New Roman}  % 设置英文局部字体
\newCJKfontfamily\zygjztcn{simhei.ttf}  % 设置中文局部字体 黑体
\newcommand{\zygjzt}[1]{\zygjztcn \zygjzten #1}  % 混合中英文
%摘要和关键字内容:
\newfontfamily\zygjzen{Times New Roman}  % 设置英文局部字体
\newCJKfontfamily\zygjzcn{simsun.ttc}  % 设置中文局部字体 宋体
\newcommand{\zygjz}[1]{\zygjzcn \zygjzen #1}  % 混合中英文
%正文内容:
\newfontfamily\zwen{Times New Roman}  % 设置英文局部字体
\newCJKfontfamily\zwcn{simsun.ttc}  % 设置中文局部字体 宋体
\newcommand{\zw}[1]{\zwcn \zwen #1}  % 混合中英文
%\setmainfont{Times New Roman}
%setCJKmainfont{SourceHanSerifSC-SemiBold.otf}
%更改全局英文字体为Times New Roman,更改中文字题为SourceHanSerifSC-SemiBold
%\newcommand{\kt}{\setCJKfamilyfont{simkai.ttf}}

%https://www.latexstudio.net/index/details/index/mid/2967.html
\definecolor{primary}{RGB}{42,83,193} %% Cerulean blue #2A53C1
\definecolor{mpibpcgreen}{RGB}{190, 214, 52} %% orange yellow #FF8E00
\definecolor{mpibpcblue}{RGB}{26, 135, 162} %% vivid green #007D34
\definecolor{mpibpcmaroon}{RGB}{176, 88, 97} %% strong violet #53377A
\definecolor{shelegreen}{RGB}{60, 122, 24} %% strong violet #53377A

\title{\color{primary} 浙江大学2022年考研真题}
\date{}
%---------------------------------------
\begin{document}
\pagenumbering{arabic}

\maketitle
\thispagestyle{fancy}
\huge
\[
\mbox{设} x_0 > 0, x_n = arctan\ x_{n-1} \quad(n = 1,2,3,\cdots)
\]
\begin{flushleft}
    \quad\qquad$\mbox{1:证明}\ \lim_{n\to\infty}x_n = 0$
    \\
    \quad\qquad$\mbox{2:证明}\ \{\sqrt{n}x_n\}\mbox{收敛,并求其极限}$
\end{flushleft}


\newpage
\pagenumbering{arabic}
\section{证明思路}
\Large
\textbf{首先我们知道$\{x_n\}$递减有界,所以必收敛}:
\\
\[arctan x < x \Rightarrow x_n < x_{n-1}\]

\textbf{其次数列$\{\sqrt{n}nx_n\}$中$\sqrt{n}$和$x_n$都单调,并且 n 趋近于$\infty$}:\\

\mbox{\boldmath \color{shelegreen}这个题目和前面我一个视频力里讲的上海交大的一道课后题非常像,}\\
\mbox{\color{shelegreen}感兴趣的朋友可以翻下上一个视频:}\\
\\
看到这个形式我们需要想到 Stolz 定理, 即:\\
设${b_n}$单调,并且 \[\lim_{n\to\infty} b_n = \infty\]
\quad 如果\[\lim_{n\to\infty} \frac{a_{n}-a_{n-1}}{b_{n}-b_{n-1}}=A \]
\quad 那么\[\lim_{n\to\infty} \frac{a_{n}}{b_{n}}=A \]

我们需要一点点小小的技巧,我们先计算数列$\{nx_n^2\}$的极限
\[nx_n^2 = \frac{n}{\frac{1}{x_n^2}}\]

\newpage
\pagenumbering{arabic}

\section{开始证明}

\[x_0 > 0, arctanx < x \Longrightarrow x_n = arctan\ x_{n-1} < x_{n-1} \]
${x_n}$单调递减,并且有界, 所以必收敛,我们设 \[\lim_{n\to\infty}x_n = a\]
我们两边同时取极限有$a = arctan a \Rightarrow a = 0$\\

\[nx_n^2 = \frac{n}{\frac{1}{x_n^2}}\]我们看看这个数列\[\{\frac{(n+1) - n}{\frac{1}{x_{n+1}^2} - \frac{1}{x_n^2}}\}\]

\[
    \frac{(n+1) - n}{\frac{1}{x_{n+1}^2} - \frac{1}{x_n^2}} = \frac{1}{\frac{1}{x_{n+1}^2} - \frac{1}{x_n^2}} = \frac{x_n^2x_{n+1}^2}{x_n^2-x_{n+1}^2} = \frac{x_n^2{(arctanx_n)}^2}{x_n^2 - (arctanx_n)^2}   
\]
考虑到 arctanx 的泰勒展开式 :
\[arctanx = x - \frac{1}{3}x^3 + o(x^3)\]
\[
\lim_{n\to\infty} \frac{x_n^2{(arctanx)}^2}{x_n^2 - (arctanx_n)^2} =  \lim_{n\to\infty} \frac{x_n^4}{x_n^2 - (x_n - \frac{1}{3}x_n^3 + o(x_n^3))^2} 
= \lim_{n\to\infty} \frac{x_n^4}{\frac{2}{3}x_n^4} = \frac{3}{2}
\]
\\

由 Stolz 定理知,$\{nx_n^2\}$收敛,并且收敛于 $\frac{3}{2}$
\\
$\qquad \quad \mbox{故}\ \{\sqrt{n}x_n\}\mbox{收敛于}\frac{\sqrt{6}}{2}$
\huge
\begin{align*}    
    \mbox{\color{mpibpcgreen} 证毕} 
\end{align*}

\end{document}  
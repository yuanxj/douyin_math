\documentclass[11pt,a4paper,oneside]{article}

\usepackage{amsmath}
\usepackage{amssymb}
\usepackage{geometry}  % 宏包]
\usepackage[UTF8]{ctex}
\usepackage{xcolor}

\geometry{left=1.8cm,right=1.8cm,top=1.5cm,bottom=2cm}
%设置页边距
\geometry{centering}
%版心水平垂直均居中,竖直居中:vcentering,水平居中:hcenterin

\usepackage{fancyhdr}

\linespread{1.25}  % 行距
\usepackage{fancyhdr}  % 宏包
\usepackage{lastpage}  % 获得总页数宏包
\pagestyle{fancy}
\fancyhf{}  % 清除页眉页脚保证默认样式不起作用

\rhead{杰哥讲数学,抖音搜索 sqrt2\_1.41421}
\usepackage{fontspec}  % 宏包



%题目
\newfontfamily\timuen{Times New Roman}   % 设置英文局部字体
\newCJKfontfamily\timucn{simhei.ttf}  % 设置中文局部字体 黑体
\newcommand{\timu}[1]{\timucn \timuen #1}  % 混合中英文
%作者:
\newfontfamily\zzen{Times New Roman}  % 设置英文局部字体
%\newCJKfontfamily\zzcn{simsong.ttf}  % 设置中文局部字体 仿宋
\newCJKfontfamily\zzcn{simfang.ttf}  % 设置中文局部字体 仿宋
\newcommand{\zz}[1]{\zzcn \zzen #1}  % 混合中英文
%单位:
\newfontfamily\dwen{Times New Roman}  % 设置英文局部字体
\newCJKfontfamily\dwcn{simsun.ttc}  % 设置中文局部字体 宋体
\newcommand{\dw}[1]{\dwcn \dwen #1}  % 混合中英文
%摘要和关键字:
\newfontfamily\zygjzten{Times New Roman}  % 设置英文局部字体
\newCJKfontfamily\zygjztcn{simhei.ttf}  % 设置中文局部字体 黑体
\newcommand{\zygjzt}[1]{\zygjztcn \zygjzten #1}  % 混合中英文
%摘要和关键字内容:
\newfontfamily\zygjzen{Times New Roman}  % 设置英文局部字体
\newCJKfontfamily\zygjzcn{simsun.ttc}  % 设置中文局部字体 宋体
\newcommand{\zygjz}[1]{\zygjzcn \zygjzen #1}  % 混合中英文
%正文内容:
\newfontfamily\zwen{Times New Roman}  % 设置英文局部字体
\newCJKfontfamily\zwcn{simsun.ttc}  % 设置中文局部字体 宋体
\newcommand{\zw}[1]{\zwcn \zwen #1}  % 混合中英文
%\setmainfont{Times New Roman}
%setCJKmainfont{SourceHanSerifSC-SemiBold.otf}
%更改全局英文字体为Times New Roman,更改中文字题为SourceHanSerifSC-SemiBold
%\newcommand{\kt}{\setCJKfamilyfont{simkai.ttf}}

\definecolor{primary}{RGB}{42,83,193} %% Cerulean blue #2A53C1
\definecolor{mpibpcgreen}{RGB}{190, 214, 52} %% orange yellow #FF8E00
\definecolor{mpibpcblue}{RGB}{26, 135, 162} %% vivid green #007D34
\definecolor{mpibpcmaroon}{RGB}{176, 88, 97} %% strong violet #53377A

\title{\color{primary} 上海交通大学数学分析课后题}
\date{}
%---------------------------------------
\begin{document}
\pagenumbering{arabic}


\maketitle
\thispagestyle{fancy}
\huge
$$
\mbox{设} x_i \in (0,1), x_{n+1} = x_n(1-x_n) (n = 1,2,3,\cdots)
$$
$$
\mbox{证明:}\{nx_n\}\mbox{收敛并求其极限}
$$

\newpage
\pagenumbering{arabic}
\section{证明思路}
\Large
\textbf{首先$\{x_n\}$递减有界,所以必收敛}:
\\
\[{x_{n+1} - x_n = - x_n^2} < 0\]

\textbf{其次$\{nx_n\}$n和$x_n$都单调,并且 n 趋近于$\infty$}:

看到这个形式我们需要想到 Stolz 定理m, 即:\\
设${b_n}$单调,并且 \[\lim_{n\to\infty} b_n = \infty\]
\quad 如果\[\lim_{n\to\infty} \frac{a_{n}-a_{n-1}}{b_{n}-b_{n-1}}=A \]
\quad 那么\[\lim_{n\to\infty} \frac{a_{n}}{b_{n}}=A \]

我们需要一点点小小的技巧
\[nx_n = \frac{n}{\frac{1}{x_n}}\]

\newpage
\pagenumbering{arabic}

\section{开始证明}
\[{x_{n+1} - x_n = - x_n^2} < 0\]
${x_n}$单调递减,并且有界, 所以必收敛,我们设 \[\lim_{n\to\infty}x_n = a\]
$x_{n+1} = x_n(1-x_n)$,我们两边同时取极限有$a = a(1-a) \Rightarrow a = 0$\\

\[nx_n = \frac{n}{\frac{1}{x_n}}\]我们看看这个数列\[\{\frac{(n+1) - n}{\frac{1}{x_{n+1}} - \frac{1}{x_n}}\}\]

\[\frac{(n+1) - n}{\frac{1}{x_{n+1}} - \frac{1}{x_n}} = \frac{1}{\frac{1}{x_{n+1}} - \frac{1}{x_n}} = \frac{x_nx_{n+1}}{x_n-x_{n+1}} = \frac{x_nx_{n+1}}{x_n^2} = \frac{x_{n+1}}{x_n} = 1-x_n\]
\[\lim_{n\to\infty} x_n = 0 \Longrightarrow \lim_{n\to\infty} (1-x_n) = 1\]
\\

由 Stolz 定理知,$\{nx_n\}$收敛,并且收敛于 1

\huge
\begin{align*}    
    \mbox{\color{mpibpcgreen} 证毕} 
\end{align*}

\end{document}  
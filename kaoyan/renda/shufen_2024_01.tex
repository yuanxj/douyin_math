\documentclass[11pt,a4paper,oneside]{article}

\usepackage{amsmath}
\usepackage{amssymb}
\usepackage{geometry}  % 宏包]
\usepackage[UTF8]{ctex}
\usepackage{xcolor}

\geometry{left=1.8cm,right=1.8cm,top=1.5cm,bottom=2cm}
%设置页边距
\geometry{centering}
%版心水平垂直均居中,竖直居中:vcentering,水平居中:hcenterin

\usepackage{fancyhdr}

\linespread{1.25}  % 行距
\usepackage{fancyhdr}  % 宏包
\usepackage{lastpage}  % 获得总页数宏包
\pagestyle{fancy}
\fancyhf{}  % 清除页眉页脚保证默认样式不起作用

\rhead{杰哥讲数学,抖音搜索 sqrt2\_1.41421}
\usepackage{fontspec}  % 宏包



%题目
\newfontfamily\timuen{Times New Roman}   % 设置英文局部字体
\newCJKfontfamily\timucn{simhei.ttf}  % 设置中文局部字体 黑体
\newcommand{\timu}[1]{\timucn \timuen #1}  % 混合中英文
%作者:
\newfontfamily\zzen{Times New Roman}  % 设置英文局部字体
%\newCJKfontfamily\zzcn{simsong.ttf}  % 设置中文局部字体 仿宋
\newCJKfontfamily\zzcn{simfang.ttf}  % 设置中文局部字体 仿宋
\newcommand{\zz}[1]{\zzcn \zzen #1}  % 混合中英文
%单位:
\newfontfamily\dwen{Times New Roman}  % 设置英文局部字体
\newCJKfontfamily\dwcn{simsun.ttc}  % 设置中文局部字体 宋体
\newcommand{\dw}[1]{\dwcn \dwen #1}  % 混合中英文
%摘要和关键字:
\newfontfamily\zygjzten{Times New Roman}  % 设置英文局部字体
\newCJKfontfamily\zygjztcn{simhei.ttf}  % 设置中文局部字体 黑体
\newcommand{\zygjzt}[1]{\zygjztcn \zygjzten #1}  % 混合中英文
%摘要和关键字内容:
\newfontfamily\zygjzen{Times New Roman}  % 设置英文局部字体
\newCJKfontfamily\zygjzcn{simsun.ttc}  % 设置中文局部字体 宋体
\newcommand{\zygjz}[1]{\zygjzcn \zygjzen #1}  % 混合中英文
%正文内容:
\newfontfamily\zwen{Times New Roman}  % 设置英文局部字体
\newCJKfontfamily\zwcn{simsun.ttc}  % 设置中文局部字体 宋体
\newcommand{\zw}[1]{\zwcn \zwen #1}  % 混合中英文
%\setmainfont{Times New Roman}
%setCJKmainfont{SourceHanSerifSC-SemiBold.otf}
%更改全局英文字体为Times New Roman,更改中文字题为SourceHanSerifSC-SemiBold
%\newcommand{\kt}{\setCJKfamilyfont{simkai.ttf}}

%https://www.latexstudio.net/index/details/index/mid/2967.html
\definecolor{primary}{RGB}{42,83,193} %% Cerulean blue #2A53C1
\definecolor{mpibpcgreen}{RGB}{190, 214, 52} %% orange yellow #FF8E00
\definecolor{mpibpcblue}{RGB}{26, 135, 162} %% vivid green #007D34
\definecolor{mpibpcmaroon}{RGB}{176, 88, 97} %% strong violet #53377A
\definecolor{shelegreen}{RGB}{60, 122, 24} %% strong violet #53377A

\title{\color{primary} 中国人民大学2024年考研真题}
\date{}
%---------------------------------------
\begin{document}
\pagenumbering{arabic}

\maketitle
\thispagestyle{fancy}
\huge
\[
\mbox{1.(15分)\ 已知 n 为正整数,求极限} \lim_{n\to\infty}\sum_{i=1}^{n}\frac{sin \frac{i\pi}{n}}{n+\frac{i}{n}}
\]


\newpage
\pagenumbering{arabic}
\section{证明思路}
\Large
\textbf{首先我们我们看到对$\frac{i}{n}$求和,我们要想到积分}:
\\
我们先忽略掉分母中的 n,显然 
\\
\[
    \lim_{n\to\infty}\sum_{i=1}^{n}\frac{1}{n}sin \frac{i\pi}{n} = \int_{0}^{1} sin\pi xdx
\]
\\
\\

\textbf{其次, 我们需要一定的放缩技巧,即:}
\[
    n < n+\frac{i}{n} < n+1
\]

\newpage

\section{开始计算}
我们先计算这个值
\[
    \lim_{n\to\infty}\sum_{i=1}^{n}\frac{1}{n}sin \frac{i\pi}{n} = \int_{0}^{1} sin\pi xdx = \frac{-cos\pi x}{\pi}|_{0}^{1} = \frac{2}{\pi}
\]
\[
    \lim_{n\to\infty}\sum_{i=1}^{n}\frac{sin \frac{i\pi}{n}}{n+\frac{i}{n}} = \lim_{n\to\infty}\sum_{i=1}^{n}\frac{n}{n+\frac{i}{n}}\frac{sin \frac{i\pi}{n}}{n} 
\]
\[
    \lim_{n\to\infty}\sum_{i=1}^{n}\frac{n}{n+1}\frac{sin \frac{i\pi}{n}}{n} < \lim_{n\to\infty}\sum_{i=1}^{n}\frac{n}{n+\frac{i}{n}}\frac{sin \frac{i\pi}{n}}{n} < \lim_{n\to\infty}\sum_{i=1}^{n}\frac{sin \frac{i\pi}{n}}{n} 
\]
\[
    \mbox{显然}\lim_{n\to\infty}\frac{n}{n+1} = 1
\]

根据迫敛定理得
\[
    \lim_{n\to\infty}\sum_{i=1}^{n}\frac{sin \frac{i\pi}{n}}{n+\frac{i}{n}} = \frac{2}{\pi}
\]


\huge
\begin{align*}    
    \mbox{\color{mpibpcgreen} 证毕} 
\end{align*}

\end{document}  
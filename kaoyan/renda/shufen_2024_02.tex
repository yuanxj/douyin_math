\documentclass[11pt,a4paper,oneside]{article}

\usepackage{amsmath}
\usepackage{amssymb}
\usepackage{geometry}  % 宏包]
\usepackage[UTF8]{ctex}
\usepackage{xcolor}

\geometry{left=1.8cm,right=1.8cm,top=1.5cm,bottom=2cm}
%设置页边距
\geometry{centering}
%版心水平垂直均居中,竖直居中:vcentering,水平居中:hcenterin

\usepackage{fancyhdr}

\linespread{1.25}  % 行距
\usepackage{fancyhdr}  % 宏包
\usepackage{lastpage}  % 获得总页数宏包
\pagestyle{fancy}
\fancyhf{}  % 清除页眉页脚保证默认样式不起作用

\rhead{杰哥讲数学,抖音搜索 sqrt2\_1.41421}
\usepackage{fontspec}  % 宏包



%题目
\newfontfamily\timuen{Times New Roman}   % 设置英文局部字体
\newCJKfontfamily\timucn{simhei.ttf}  % 设置中文局部字体 黑体
\newcommand{\timu}[1]{\timucn \timuen #1}  % 混合中英文
%作者:
\newfontfamily\zzen{Times New Roman}  % 设置英文局部字体
%\newCJKfontfamily\zzcn{simsong.ttf}  % 设置中文局部字体 仿宋
\newCJKfontfamily\zzcn{simfang.ttf}  % 设置中文局部字体 仿宋
\newcommand{\zz}[1]{\zzcn \zzen #1}  % 混合中英文
%单位:
\newfontfamily\dwen{Times New Roman}  % 设置英文局部字体
\newCJKfontfamily\dwcn{simsun.ttc}  % 设置中文局部字体 宋体
\newcommand{\dw}[1]{\dwcn \dwen #1}  % 混合中英文
%摘要和关键字:
\newfontfamily\zygjzten{Times New Roman}  % 设置英文局部字体
\newCJKfontfamily\zygjztcn{simhei.ttf}  % 设置中文局部字体 黑体
\newcommand{\zygjzt}[1]{\zygjztcn \zygjzten #1}  % 混合中英文
%摘要和关键字内容:
\newfontfamily\zygjzen{Times New Roman}  % 设置英文局部字体
\newCJKfontfamily\zygjzcn{simsun.ttc}  % 设置中文局部字体 宋体
\newcommand{\zygjz}[1]{\zygjzcn \zygjzen #1}  % 混合中英文
%正文内容:
\newfontfamily\zwen{Times New Roman}  % 设置英文局部字体
\newCJKfontfamily\zwcn{simsun.ttc}  % 设置中文局部字体 宋体
\newcommand{\zw}[1]{\zwcn \zwen #1}  % 混合中英文
%\setmainfont{Times New Roman}
%setCJKmainfont{SourceHanSerifSC-SemiBold.otf}
%更改全局英文字体为Times New Roman,更改中文字题为SourceHanSerifSC-SemiBold
%\newcommand{\kt}{\setCJKfamilyfont{simkai.ttf}}

%https://www.latexstudio.net/index/details/index/mid/2967.html
\definecolor{primary}{RGB}{42,83,193} %% Cerulean blue #2A53C1
\definecolor{mpibpcgreen}{RGB}{190, 214, 52} %% orange yellow #FF8E00
\definecolor{mpibpcblue}{RGB}{26, 135, 162} %% vivid green #007D34
\definecolor{mpibpcmaroon}{RGB}{176, 88, 97} %% strong violet #53377A
\definecolor{shelegreen}{RGB}{60, 122, 24} %% strong violet #53377A

\title{\color{primary} 中国人民大学2024年考研真题}
\date{}
%---------------------------------------
\begin{document}
\pagenumbering{arabic}

\maketitle
\thispagestyle{fancy}
\huge
\[
\mbox{2.(15分)\ 设f(x)在} [0,\infty) \\ \mbox{上有定义,并且一致连续,已知}
\]
\[
    \lim_{n\to\infty} f(x+n) = 0\mbox{(n为自然数), 对}\forall x > 0 
\]

\begin{flushleft}
    \quad$\mbox{证明:}\ \lim_{x\to\infty}f(x) = 0$
\end{flushleft}


\newpage
\pagenumbering{arabic}
\section{证明思路}
\Large
\textbf{首先我们用$\epsilon, \delta$语言来看下什么是一致连续和极限}:
\\
f(x)一致连续, 表示$\forall \epsilon > 0,\ \exists \delta,\ \forall |x_1 - x_2| < \delta,\ |f(x_1) - f(x_2)| < \epsilon$
\\

\textbf{其次, 我们取 x=1, 数列${f(N)}$显然收敛于 0}


\mbox{从这里我们可以自然的想到,把(0,1)区间按照$\delta$拆成 k 段:}\\

对任意的一个区间$(a_1, a_2), |a_1-a_2| < \delta, \exists N_1, \forall n>N_1, f(n+a_1) < \epsilon$ 
那么我们取$N=max(\{N_i\}), i=1,2,3,\cdots,k$
\\
$\qquad\ \forall x>N, f(x) = f([x] + a_i + \delta_1)$ 其中$[x]>n, \delta_1 < \delta$
\\
显然,\ $|f([x] + a_i + \delta_1) - f([x]+a_i)| < \epsilon$
\\
到这里基本上证明已经走通了
\newpage
\pagenumbering{arabic}

\section{开始证明}

f(x)一致连续
\[
    \forall \epsilon > 0, \exists \delta > 0, \forall |x_1-x_2| < \delta\Rightarrow |f(x_1)-f(x_2)| < \frac{\epsilon}{2} 
\]
我们取$k = [\frac{1}{\delta}] + 1$
令 $a_i = \frac{i}{k}, i=1,2,3,\cdots,k-1$ 
\\
显然对$\forall a_i, \exists N_i, \forall n>N_i, f(a_i+n) < \frac{\epsilon}{2}$
\\
我们取$N = max({N_i}),  \forall x>N, x=[x]+a_i+\delta_1, [x]>N,\delta_1<\delta,\  \\
  f(x) = f([x]+a_i+\delta_1) < |f([x]+a_i+\delta_1) - f([x]+a_i)| + |f([x]+a_i)| < \frac{\epsilon}{2} + \frac{\epsilon}{2} = \epsilon$

\huge
\begin{align*}    
    \mbox{\color{mpibpcgreen} 证毕} 
\end{align*}

\end{document}  
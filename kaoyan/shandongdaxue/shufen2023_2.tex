\documentclass{ctexart}
% Required package
\usepackage[UTF8]{ctex}
\usepackage{amsmath,amssymb,amsfonts}
\usepackage{bm}
\usepackage[dvipsnames]{xcolor}

\begin{document}
\fontsize{25.0pt}{\baselineskip}\selectfont

\begin{center}
    \color{ForestGreen} 2023年山东大学数学分析真题
\end{center}

\fontsize{25.0pt}{\baselineskip}\selectfont
\begin{align*}
&\mbox{(15分)}\quad a_n>0, \lim_{n\to\infty}sup{\sqrt[n]{a_n}} <= 1, \\
&\quad\quad\mbox{证明:}\forall l>1, \lim_{n\to\infty}\frac{a_n}{l^n} = 0
\end{align*}

\pagenumbering{arabic}
\newpage
\pagenumbering{arabic}
\fontsize{20.0pt}{\baselineskip}\selectfont

\begin{align*}
&\mbox{我们仅需证明,对}\forall l > 1, \forall \epsilon>0, \exists N, \forall n > N , \frac{a_n}{l^n} < \epsilon &\\
\\
&\lim_{n\to\infty}sup{\sqrt[n]{a_n}} <= 1 &\\
&\Longrightarrow \exists N_1, \forall n>N_1, \sqrt[n]{a_n} < 1+\frac{l-1}{2}\\
\\
&\mbox{显然} \quad\quad 0<1+\frac{l-1}{2}<l &\\
&\Longrightarrow \lim_{n\to\infty} (\frac{1+\frac{l-1}{2}}{l})^n = 0&\\
&\Longrightarrow \exists N_2, \forall n>N_2, (\frac{1+\frac{l-1}{2}}{l})^n < \epsilon &\\
&\mbox{取} N = max\{N_1, N_2\}, \forall n>N, &\\
&\frac{a_n}{l^n} = (\frac{\sqrt[n]{a_n}}{l})^n \le (\frac{1+\frac{l-1}{2}}{l})^n < \epsilon &\\
\\
&\mbox{证毕}&
\end{align*}

\end{document}